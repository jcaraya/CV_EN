%%%%%%%%%%%%%%%%%%%%%%%%%%%%%%%%%%%%%%%%%
% Friggeri Resume/CV
% XeLaTeX Template
% Version 1.2 (3/5/15)
%
% This template has been downloaded from:
% http://www.LaTeXTemplates.com
%
% Original author:
% Adrien Friggeri (adrien@friggeri.net)
% https://github.com/afriggeri/CV
%
% License:
% CC BY-NC-SA 3.0 (http://creativecommons.org/licenses/by-nc-sa/3.0/)
%
% Important notes:
% This template needs to be compiled with XeLaTeX and the bibliography, if used,
% needs to be compiled with biber rather than bibtex.
%
%%%%%%%%%%%%%%%%%%%%%%%%%%%%%%%%%%%%%%%%%

\documentclass[]{friggeri-cv} % Add 'print' as an option into the square bracket to remove colors from this template for printing

\begin{document}

\header{Jorge}{Carvajal}{Computer Engineer} % Your name and current job title/field

%----------------------------------------------------------------------------------------
%	SIDEBAR SECTION
%----------------------------------------------------------------------------------------

\begin{aside} % In the aside, each new line forces a line break
\section{contact}
ID: 702230223
Marital status: single
Birthdate: 15/10/1993
~
Address:
San Joaquín, Heredia
Costa Rica
~
+(506) 8939 0091
jorgeart15@gmail.com
~
\section{technical skills}
OS:
Windows, Linux
~
Languages: 
C/C++, Python
~
Deep Learning:
Keras, Tensorflow
~
Data Analysis:
Jupyter notebook, Numpy, Pandas, Matplotlib
~
Other: 
Git, LaTex, Vim
\end{aside}


%----------------------------------------------------------------------------------------
%	SUMMARY SECTION
%----------------------------------------------------------------------------------------

\section{summary}
Analytical, dedicated, and responsible engineer with 2+ years of experience developing robust code for high performance networking equipment using C/C++ as main programming languages and python for tasks related to testing, process automation and data analysis.

My objective is to work in a competitive environment, on challenging and innovative assignments while receiving constant feedback and having opportunities for personal and professional development.

%----------------------------------------------------------------------------------------
%	EDUCATION SECTION
%----------------------------------------------------------------------------------------

\section{education}
\begin{entrylist}

\entry
{2017–Now}
{Deep Learning Nanodegree}
{Udacity}
{Four month term in progress}

\entry
{Jan 2018}
{TOEFL IBT {\normalfont - Score: 99}}
{ETS}
{Reading: 25, Listening: 25, Speaking: 23, Writing: 26}

\entry
{2011--2016}
{Bachelor's Degree in Computer Engineering}
{Costa Rica Institue of Technology}
{Graduate with honors. Score: 90.95}

\end{entrylist}

%----------------------------------------------------------------------------------------
%	WORK EXPERIENCE SECTION
%----------------------------------------------------------------------------------------

\section{experience}


\begin{entrylist}

%------------------------------------------------

\entry
{2017--Now}
{Software Engineer}
{Aruba, a Hewlett Packard Enterprise Company}
{Software developer of the Aruba OS CX, which is used for Aruba's first core switch (Aruba 8400). Participated in the implementation of features related to high availability, linecard hotswap and multicast.} 

\entry
{2016--2017}
{Software Engineer Intern}
{Aruba, a Hewlett Packard Enterprise Company}
{Non-cryptographic hashing algorithm evaluation for the Aruba OS CX
in terms of processing speed, collision resistance and distribution in the available hash table buckets.} 

\entry
{2015--2016}
{Student Exchange Program: Digital Integrated Circuits Course}
{ITESM, Mexico}
{Design and development of a communication system that converts 8 bit parallel data into a syncronous serial signal. The project was implemented using LTSpice integrated with Electric VLSI and following the MOSIS submicron design rules to allow its further manufacturing process.} 

\entry
{2015--2016}
{Student Exchange Internship: Scaffold manufacturing for cell culture}
{ITESM, Mexico}
{Integration of a programmable power source and positive displacement pump to the manufacturing process of 3D scaffolds for cellular culture. This required to understand the operation of the new equipment, design an interface that allows to control it using NI LabView and integrate the system in the production line.} 

%------------------------------------------------

\end{entrylist}
\end{document}